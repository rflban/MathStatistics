\chapter{Практическая часть}

\lstset{language=matlab}

В листинге~\ref{lst:programm} приведён текст программы.
\begin{lstlisting}[caption={Текст программы},label={lst:programm}]
function lab1()
    X = [
        -13.40, -12.63, -13.65, -14.23, -13.39, -12.36, ...
        -13.52, -13.44, -13.87, -11.82, -12.01, -11.40, ...
        -13.02, -12.61, -13.06, -13.75, -13.55, -14.01, ...
        -11.75, -12.95, -12.59, -13.60, -12.76, -11.05, ...
        -13.15, -13.61, -11.73, -13.00, -12.66, -12.67, ...
        -12.60, -12.47, -13.52, -12.61, -11.93, -13.11, ...
        -13.22, -11.87, -13.44, -12.70, -11.78, -12.30, ...
        -12.89, -13.29, -12.48, -10.44, -12.55, -12.64, ...
        -12.03, -14.60, -14.56, -13.30, -11.32, -12.24, ...
        -11.17, -12.50, -13.25, -12.55, -12.85, -12.67, ...
        -12.41, -12.58, -12.10, -13.54, -12.69, -12.87, ...
        -12.71, -12.77, -13.30, -12.74, -12.73, -12.64, ...
        -12.18, -11.20, -12.40, -13.78, -13.71, -10.74, ...
        -11.89, -13.20, -11.31, -14.26, -10.38, -12.88, ...
        -11.39, -11.35, -12.55, -12.84, -10.25, -12.40, ...
        -14.01, -11.47, -13.14, -12.69, -11.92, -12.86, ...
        -13.06, -12.57, -13.63, -12.34, -12.84, -14.03, ...
        -13.34, -11.64, -13.58, -10.44, -11.37, -11.01, ...
        -13.80, -13.27, -12.32, -10.69, -12.92, -13.29, ...
        -12.58, -13.98, -11.46, -11.82, -12.33, -11.47, ...
    ];
    X = sort(X);

    % Максимальное значение выборки
    M_max = max(X);
    % Минимальное значение выборки
    M_min = min(X);

    % Разброс выборки
    R = M_max - M_min;

    % Оценка мат. ожидания
    M = mean(X);
    % Оценка дисперсии
    D = var(X);

    fprintf('M_max    = %9.5f\n', M_max);
    fprintf('M_min    = %9.5f\n', M_min);
    fprintf('R        = %9.5f\n', R);
    fprintf('mean     = %9.5f\n', M);
    fprintf('varience = %9.5f\n', D);
    fprintf('\n');

    m = floor(log2(length(X))) + 2;
    % Группировка значений в m интервалов
    [N, edges] = histcounts(X, m, 'BinLimits', [M_min, M_max]);

    % Вывод всех интервалов, кроме последнего,
    % потому что последний - это отрезок,
    % а все остальные - полуинтервалы
    fprintf('%d intervals:\n', m);
    for i = 1 : (length(N) - 1)
        fprintf('%3d values in [%f,%f)\n', ...
                N(i), edges(i), edges(i + 1));
    end
    % Вывод последнего отрезка
    fprintf('%3d values in [%f,%f]\n', ...
            N(end), edges(end - 1), edges(end));

    f = normpdf(X, M, sqrt(D));
    F = normcdf(X, M, sqrt(D));

    subplot(2, 1, 1);
    % Построение гистограммы
    histogram(X, m, 'Normalization', 'pdf', ...
              'BinLimits', [M_min, M_max]);
    hold on;
    % Построение графика функции плотности распределения
    plot(X, f, 'LineWidth', 2);
    hold off;

    subplot(2, 1, 2);
    % Построение графика эмпирической функции распределения
    [YY, XX] = ecdf(X);
    stairs(XX, YY, 'LineWidth', 2);
    hold on;
    % Построение графика функции распределения
    plot(X, F, 'LineWidth', 2);
    hold off;
end
\end{lstlisting}

В листинге~\ref{lst:result} приведён результат выполнения описанной программы.
\begin{lstlisting}[language=,numbers=none,caption={Результат программы},label={lst:result}]
M_max    = -10.25000
M_min    = -14.60000
R        =   4.35000
mean     = -12.61483
varience =   0.86533

8 intervals:
  4 values in [-14.600000,-14.056250)
 18 values in [-14.056250,-13.512500)
 20 values in [-13.512500,-12.968750)
 36 values in [-12.968750,-12.425000)
 16 values in [-12.425000,-11.881250)
 14 values in [-11.881250,-11.337500)
  6 values in [-11.337500,-10.793750)
  6 values in [-10.793750,-10.250000]
\end{lstlisting}

\begin{figure}[H]
    \caption{Гистограмма и график функции плотности распределения вероятностей нормальной случайной величины с математическим ожиданием $\hat \mu$ и дисперсией $S^2$}\label{img:plot01}

    % This file was created by matlab2tikz.
%
%The latest updates can be retrieved from
%  http://www.mathworks.com/matlabcentral/fileexchange/22022-matlab2tikz-matlab2tikz
%where you can also make suggestions and rate matlab2tikz.
%
\definecolor{mycolor1}{rgb}{0.00000,0.44700,0.74100}%
\definecolor{mycolor2}{rgb}{0.85000,0.32500,0.09800}%
\definecolor{mycolor3}{rgb}{0.92900,0.69400,0.12500}%
\definecolor{mycolor4}{rgb}{0.49400,0.18400,0.55600}%
%
\begin{tikzpicture}[scale=0.8]

\begin{axis}[%
width=7.266in,
height=6.376in,
at={(1.219in,0.861in)},
scale only axis,
unbounded coords=jump,
xmin=0,
xmax=120,
xlabel style={font=\color{white!15!black}},
xlabel={n},
ymin=-15.5,
ymax=-10.5,
ylabel style={font=\color{white!15!black}},
ylabel={y},
axis background/.style={fill=white},
legend style={legend cell align=left, align=left, draw=white!15!black}
]
\addplot [color=mycolor1]
  table[row sep=crcr]{%
1	-12.6148333333333\\
2	-12.6148333333333\\
3	-12.6148333333333\\
4	-12.6148333333333\\
5	-12.6148333333333\\
6	-12.6148333333333\\
7	-12.6148333333333\\
8	-12.6148333333333\\
9	-12.6148333333333\\
10	-12.6148333333333\\
11	-12.6148333333333\\
12	-12.6148333333333\\
13	-12.6148333333333\\
14	-12.6148333333333\\
15	-12.6148333333333\\
16	-12.6148333333333\\
17	-12.6148333333333\\
18	-12.6148333333333\\
19	-12.6148333333333\\
20	-12.6148333333333\\
21	-12.6148333333333\\
22	-12.6148333333333\\
23	-12.6148333333333\\
24	-12.6148333333333\\
25	-12.6148333333333\\
26	-12.6148333333333\\
27	-12.6148333333333\\
28	-12.6148333333333\\
29	-12.6148333333333\\
30	-12.6148333333333\\
31	-12.6148333333333\\
32	-12.6148333333333\\
33	-12.6148333333333\\
34	-12.6148333333333\\
35	-12.6148333333333\\
36	-12.6148333333333\\
37	-12.6148333333333\\
38	-12.6148333333333\\
39	-12.6148333333333\\
40	-12.6148333333333\\
41	-12.6148333333333\\
42	-12.6148333333333\\
43	-12.6148333333333\\
44	-12.6148333333333\\
45	-12.6148333333333\\
46	-12.6148333333333\\
47	-12.6148333333333\\
48	-12.6148333333333\\
49	-12.6148333333333\\
50	-12.6148333333333\\
51	-12.6148333333333\\
52	-12.6148333333333\\
53	-12.6148333333333\\
54	-12.6148333333333\\
55	-12.6148333333333\\
56	-12.6148333333333\\
57	-12.6148333333333\\
58	-12.6148333333333\\
59	-12.6148333333333\\
60	-12.6148333333333\\
61	-12.6148333333333\\
62	-12.6148333333333\\
63	-12.6148333333333\\
64	-12.6148333333333\\
65	-12.6148333333333\\
66	-12.6148333333333\\
67	-12.6148333333333\\
68	-12.6148333333333\\
69	-12.6148333333333\\
70	-12.6148333333333\\
71	-12.6148333333333\\
72	-12.6148333333333\\
73	-12.6148333333333\\
74	-12.6148333333333\\
75	-12.6148333333333\\
76	-12.6148333333333\\
77	-12.6148333333333\\
78	-12.6148333333333\\
79	-12.6148333333333\\
80	-12.6148333333333\\
81	-12.6148333333333\\
82	-12.6148333333333\\
83	-12.6148333333333\\
84	-12.6148333333333\\
85	-12.6148333333333\\
86	-12.6148333333333\\
87	-12.6148333333333\\
88	-12.6148333333333\\
89	-12.6148333333333\\
90	-12.6148333333333\\
91	-12.6148333333333\\
92	-12.6148333333333\\
93	-12.6148333333333\\
94	-12.6148333333333\\
95	-12.6148333333333\\
96	-12.6148333333333\\
97	-12.6148333333333\\
98	-12.6148333333333\\
99	-12.6148333333333\\
100	-12.6148333333333\\
101	-12.6148333333333\\
102	-12.6148333333333\\
103	-12.6148333333333\\
104	-12.6148333333333\\
105	-12.6148333333333\\
106	-12.6148333333333\\
107	-12.6148333333333\\
108	-12.6148333333333\\
109	-12.6148333333333\\
110	-12.6148333333333\\
111	-12.6148333333333\\
112	-12.6148333333333\\
113	-12.6148333333333\\
114	-12.6148333333333\\
115	-12.6148333333333\\
116	-12.6148333333333\\
117	-12.6148333333333\\
118	-12.6148333333333\\
119	-12.6148333333333\\
120	-12.6148333333333\\
};
\addlegendentry{$\hat \mu(\vec x_N)$}

\addplot [color=mycolor2]
  table[row sep=crcr]{%
1	-13.4\\
2	-13.015\\
3	-13.2266666666667\\
4	-13.4775\\
5	-13.46\\
6	-13.2766666666667\\
7	-13.3114285714286\\
8	-13.3275\\
9	-13.3877777777778\\
10	-13.231\\
11	-13.12\\
12	-12.9766666666667\\
13	-12.98\\
14	-12.9535714285714\\
15	-12.9606666666667\\
16	-13.01\\
17	-13.0417647058824\\
18	-13.0955555555556\\
19	-13.0247368421053\\
20	-13.021\\
21	-13.0004761904762\\
22	-13.0277272727273\\
23	-13.0160869565217\\
24	-12.9341666666667\\
25	-12.9428\\
26	-12.9684615384615\\
27	-12.9225925925926\\
28	-12.9253571428571\\
29	-12.9162068965517\\
30	-12.908\\
31	-12.898064516129\\
32	-12.8846875\\
33	-12.9039393939394\\
34	-12.8952941176471\\
35	-12.8677142857143\\
36	-12.8744444444444\\
37	-12.8837837837838\\
38	-12.8571052631579\\
39	-12.8720512820513\\
40	-12.86775\\
41	-12.8412195121951\\
42	-12.8283333333333\\
43	-12.8297674418605\\
44	-12.8402272727273\\
45	-12.8322222222222\\
46	-12.7802173913043\\
47	-12.7753191489362\\
48	-12.7725\\
49	-12.7573469387755\\
50	-12.7942\\
51	-12.8288235294118\\
52	-12.8378846153846\\
53	-12.8092452830189\\
54	-12.7987037037037\\
55	-12.7690909090909\\
56	-12.7642857142857\\
57	-12.7728070175439\\
58	-12.7689655172414\\
59	-12.7703389830508\\
60	-12.7686666666667\\
61	-12.7627868852459\\
62	-12.7598387096774\\
63	-12.7493650793651\\
64	-12.76171875\\
65	-12.7606153846154\\
66	-12.7622727272727\\
67	-12.7614925373134\\
68	-12.7616176470588\\
69	-12.7694202898551\\
70	-12.769\\
71	-12.7684507042254\\
72	-12.7666666666667\\
73	-12.7586301369863\\
74	-12.7375675675676\\
75	-12.7330666666667\\
76	-12.7468421052632\\
77	-12.7593506493506\\
78	-12.7334615384615\\
79	-12.7227848101266\\
80	-12.72875\\
81	-12.7112345679012\\
82	-12.7301219512195\\
83	-12.7018072289157\\
84	-12.7039285714286\\
85	-12.6884705882353\\
86	-12.6729069767442\\
87	-12.6714942528736\\
88	-12.6734090909091\\
89	-12.6461797752809\\
90	-12.6434444444444\\
91	-12.6584615384615\\
92	-12.6455434782609\\
93	-12.6508602150538\\
94	-12.6512765957447\\
95	-12.6435789473684\\
96	-12.6458333333333\\
97	-12.6501030927835\\
98	-12.6492857142857\\
99	-12.6591919191919\\
100	-12.656\\
101	-12.6578217821782\\
102	-12.6712745098039\\
103	-12.6777669902913\\
104	-12.6677884615385\\
105	-12.6764761904762\\
106	-12.6553773584906\\
107	-12.6433644859813\\
108	-12.6282407407407\\
109	-12.6389908256881\\
110	-12.6447272727273\\
111	-12.6418018018018\\
112	-12.624375\\
113	-12.6269911504425\\
114	-12.6328070175439\\
115	-12.632347826087\\
116	-12.6439655172414\\
117	-12.6338461538462\\
118	-12.6269491525424\\
119	-12.6244537815126\\
120	-12.6148333333333\\
};
\addlegendentry{$\hat \mu(\vec x_n)$}

\addplot [color=mycolor3]
  table[row sep=crcr]{%
1	nan\\
2	-15.4457943331499\\
3	-14.122922034411\\
4	-14.2581036510353\\
5	-14.0090074627681\\
6	-13.8388053742616\\
7	-13.7745307877199\\
8	-13.7197109478573\\
9	-13.7453048410365\\
10	-13.657576491049\\
11	-13.5513020502332\\
12	-13.4440587185114\\
13	-13.4067243035281\\
14	-13.3489048289485\\
15	-13.3269153956006\\
16	-13.3617841100821\\
17	-13.3754974782451\\
18	-13.4227373730813\\
19	-13.3567777745581\\
20	-13.3351724351917\\
21	-13.3006453060128\\
22	-13.3170923762514\\
23	-13.2927296957165\\
24	-13.2334973130441\\
25	-13.2297874351381\\
26	-13.247216282124\\
27	-13.201620668907\\
28	-13.1939093862221\\
29	-13.1754727610509\\
30	-13.1585678868195\\
31	-13.1407352596752\\
32	-13.120504531647\\
33	-13.1345959416215\\
34	-13.1193401577211\\
35	-13.0900267318628\\
36	-13.0906198626841\\
37	-13.094467682129\\
38	-13.0669084417912\\
39	-13.0778148643385\\
40	-13.0683061194592\\
41	-13.041743539673\\
42	-13.025110265838\\
43	-13.0218231406251\\
44	-13.0285914614891\\
45	-13.016751729802\\
46	-12.9806357834213\\
47	-12.9715097843614\\
48	-12.9645341693049\\
49	-12.9470526227887\\
50	-12.9900021901598\\
51	-13.0292571583372\\
52	-13.0349387949334\\
53	-13.0083403173718\\
54	-12.9948058746472\\
55	-12.9678101659341\\
56	-12.9595278243699\\
57	-12.9650635722839\\
58	-12.9579316875696\\
59	-12.9560360268428\\
60	-12.9512137076253\\
61	-12.9425373010317\\
62	-12.9366892772892\\
63	-12.9242194008486\\
64	-12.9350094193454\\
65	-12.9311893813541\\
66	-12.9302272112614\\
67	-12.9268901355023\\
68	-12.9245302928572\\
69	-12.9304479696283\\
70	-12.9276803982854\\
71	-12.924852476916\\
72	-12.9208804340576\\
73	-12.9112773128577\\
74	-12.8921464296182\\
75	-12.8857282691446\\
76	-12.8991934247385\\
77	-12.9111216471378\\
78	-12.8893282520328\\
79	-12.8776649628208\\
80	-12.8819806127954\\
81	-12.8653211768711\\
82	-12.8855069998278\\
83	-12.8623532448768\\
84	-12.8625692255745\\
85	-12.8473072323506\\
86	-12.8319848424466\\
87	-12.8287304094878\\
88	-12.8288613515662\\
89	-12.8063842817012\\
90	-12.8019052027112\\
91	-12.8171281750475\\
92	-12.8039196576744\\
93	-12.8077555656303\\
94	-12.8064784389524\\
95	-12.7976535337656\\
96	-12.7983245409726\\
97	-12.8011651063775\\
98	-12.7987894774642\\
99	-12.8080748187508\\
100	-12.8034676104705\\
101	-12.8038395552025\\
102	-12.8175550725968\\
103	-12.8230076344874\\
104	-12.8125634555122\\
105	-12.8205763952403\\
106	-12.8023315376583\\
107	-12.7902851000628\\
108	-12.7759300445886\\
109	-12.7863900907804\\
110	-12.7910785382062\\
111	-12.7868984690202\\
112	-12.7710360931066\\
113	-12.772402394663\\
114	-12.7772488433844\\
115	-12.7755197707912\\
116	-12.7871892681772\\
117	-12.7768185295597\\
118	-12.7691554720445\\
119	-12.7655111683739\\
120	-12.7556073217038\\
};
\addlegendentry{$\underline{\mu}(\vec x_n)$}

\addplot [color=mycolor4]
  table[row sep=crcr]{%
1	nan\\
2	-10.5842056668501\\
3	-12.3304112989223\\
4	-12.6968963489646\\
5	-12.9109925372319\\
6	-12.7145279590718\\
7	-12.8483263551372\\
8	-12.9352890521427\\
9	-13.030250714519\\
10	-12.804423508951\\
11	-12.6886979497668\\
12	-12.5092746148219\\
13	-12.5532756964719\\
14	-12.5582380281944\\
15	-12.5944179377328\\
16	-12.6582158899179\\
17	-12.7080319335196\\
18	-12.7683737380298\\
19	-12.6926959096524\\
20	-12.7068275648083\\
21	-12.7003070749396\\
22	-12.7383621692031\\
23	-12.7394442173269\\
24	-12.6348360202893\\
25	-12.6558125648619\\
26	-12.6897067947991\\
27	-12.6435645162782\\
28	-12.6568048994922\\
29	-12.6569410320526\\
30	-12.6574321131805\\
31	-12.6553937725829\\
32	-12.648870468353\\
33	-12.6732828462573\\
34	-12.671248077573\\
35	-12.6454018395658\\
36	-12.6582690262048\\
37	-12.6730998854386\\
38	-12.6473020845246\\
39	-12.6662876997641\\
40	-12.6671938805408\\
41	-12.6406954847173\\
42	-12.6315564008287\\
43	-12.6377117430959\\
44	-12.6518630839654\\
45	-12.6476927146424\\
46	-12.5797989991874\\
47	-12.5791285135109\\
48	-12.5804658306951\\
49	-12.5676412547623\\
50	-12.5983978098402\\
51	-12.6283899004864\\
52	-12.6408304358358\\
53	-12.6101502486659\\
54	-12.6026015327602\\
55	-12.5703716522477\\
56	-12.5690436042016\\
57	-12.5805504628038\\
58	-12.5799993469131\\
59	-12.5846419392589\\
60	-12.586119625708\\
61	-12.5830364694601\\
62	-12.5829881420656\\
63	-12.5745107578816\\
64	-12.5884280806546\\
65	-12.5900413878767\\
66	-12.594318243284\\
67	-12.5960949391245\\
68	-12.5987050012604\\
69	-12.6083926100819\\
70	-12.6103196017146\\
71	-12.6120489315347\\
72	-12.6124528992757\\
73	-12.6059829611149\\
74	-12.582988705517\\
75	-12.5804050641887\\
76	-12.5944907857878\\
77	-12.6075796515635\\
78	-12.5775948248903\\
79	-12.5679046574323\\
80	-12.5755193872046\\
81	-12.5571479589314\\
82	-12.5747369026112\\
83	-12.5412612129545\\
84	-12.5452879172826\\
85	-12.52963394412\\
86	-12.5138291110417\\
87	-12.5142580962593\\
88	-12.5179568302519\\
89	-12.4859752688606\\
90	-12.4849836861776\\
91	-12.4997949018756\\
92	-12.4871672988473\\
93	-12.4939648644772\\
94	-12.4960747525369\\
95	-12.4895043609713\\
96	-12.4933421256941\\
97	-12.4990410791895\\
98	-12.4997819511073\\
99	-12.5103090196331\\
100	-12.5085323895295\\
101	-12.5118040091539\\
102	-12.5249939470111\\
103	-12.5325263460951\\
104	-12.5230134675647\\
105	-12.5323759857121\\
106	-12.5084231793229\\
107	-12.4964438718998\\
108	-12.4805514368929\\
109	-12.4915915605957\\
110	-12.4983760072483\\
111	-12.4967051345834\\
112	-12.4777139068934\\
113	-12.4815799062219\\
114	-12.4883651917033\\
115	-12.4891758813827\\
116	-12.5007417663055\\
117	-12.4908737781326\\
118	-12.4847428330402\\
119	-12.4833963946513\\
120	-12.4740593449628\\
};
\addlegendentry{$\overline{\mu}(\vec x_n)$}

\end{axis}
\end{tikzpicture}%

\end{figure}

\begin{figure}[H]
    \caption{Графики эмпирической функции распределения и функции распределения нормальной случайной величины с математическим ожиданием $\hat \mu$ и дисперсией $S^2$}\label{img:plot02}
    % This file was created by matlab2tikz.
%
%The latest updates can be retrieved from
%  http://www.mathworks.com/matlabcentral/fileexchange/22022-matlab2tikz-matlab2tikz
%where you can also make suggestions and rate matlab2tikz.
%
\definecolor{mycolor1}{rgb}{0.00000,0.44700,0.74100}%
\definecolor{mycolor2}{rgb}{0.85000,0.32500,0.09800}%
%
\begin{tikzpicture}[scale=.85]

\begin{axis}[%
width=7.007in,
height=3.401in,
at={(1.175in,5.82in)},
scale only axis,
xmin=-15,
xmax=-10,
ymin=0,
ymax=1,
axis background/.style={fill=white},
legend style={at={(0.03,0.97)}, anchor=north west, legend cell align=left, align=left, draw=white!15!black}
]
\addplot[const plot, color=mycolor1, line width=2.0pt] table[row sep=crcr] {%
-14.6	0\\
-14.6	0.0083333333333333\\
-14.56	0.0166666666666666\\
-14.26	0.0249999999999999\\
-14.23	0.0333333333333332\\
-14.03	0.0416666666666665\\
-14.01	0.0583333333333332\\
-13.98	0.0666666666666665\\
-13.87	0.0749999999999998\\
-13.8	0.0833333333333331\\
-13.78	0.0916666666666665\\
-13.75	0.0999999999999998\\
-13.71	0.108333333333333\\
-13.65	0.116666666666666\\
-13.63	0.125\\
-13.61	0.133333333333333\\
-13.6	0.141666666666666\\
-13.58	0.15\\
-13.55	0.158333333333333\\
-13.54	0.166666666666666\\
-13.52	0.183333333333333\\
-13.44	0.2\\
-13.4	0.208333333333333\\
-13.39	0.216666666666666\\
-13.34	0.225\\
-13.3	0.241666666666666\\
-13.29	0.258333333333333\\
-13.27	0.266666666666666\\
-13.25	0.275\\
-13.22	0.283333333333333\\
-13.2	0.291666666666666\\
-13.15	0.3\\
-13.14	0.308333333333333\\
-13.11	0.316666666666666\\
-13.06	0.333333333333333\\
-13.02	0.341666666666666\\
-13	0.35\\
-12.95	0.358333333333333\\
-12.92	0.366666666666666\\
-12.89	0.375\\
-12.88	0.383333333333333\\
-12.87	0.391666666666666\\
-12.86	0.4\\
-12.85	0.408333333333333\\
-12.84	0.425\\
-12.77	0.433333333333333\\
-12.76	0.441666666666666\\
-12.74	0.45\\
-12.73	0.458333333333333\\
-12.71	0.466666666666666\\
-12.7	0.475\\
-12.69	0.491666666666666\\
-12.67	0.508333333333333\\
-12.66	0.516666666666666\\
-12.64	0.533333333333333\\
-12.63	0.541666666666666\\
-12.61	0.558333333333333\\
-12.6	0.566666666666666\\
-12.59	0.575\\
-12.58	0.591666666666666\\
-12.57	0.6\\
-12.55	0.625\\
-12.5	0.633333333333333\\
-12.48	0.641666666666666\\
-12.47	0.65\\
-12.41	0.658333333333333\\
-12.4	0.675\\
-12.36	0.683333333333333\\
-12.34	0.691666666666666\\
-12.33	0.7\\
-12.32	0.708333333333333\\
-12.3	0.716666666666667\\
-12.24	0.725\\
-12.18	0.733333333333333\\
-12.1	0.741666666666666\\
-12.03	0.75\\
-12.01	0.758333333333333\\
-11.93	0.766666666666666\\
-11.92	0.775\\
-11.89	0.783333333333333\\
-11.87	0.791666666666667\\
-11.82	0.808333333333333\\
-11.78	0.816666666666666\\
-11.75	0.825\\
-11.73	0.833333333333333\\
-11.64	0.841666666666667\\
-11.47	0.858333333333333\\
-11.46	0.866666666666667\\
-11.4	0.875\\
-11.39	0.883333333333333\\
-11.37	0.891666666666667\\
-11.35	0.9\\
-11.32	0.908333333333333\\
-11.31	0.916666666666667\\
-11.2	0.925\\
-11.17	0.933333333333333\\
-11.05	0.941666666666667\\
-11.01	0.95\\
-10.74	0.958333333333333\\
-10.69	0.966666666666667\\
-10.44	0.983333333333333\\
-10.38	0.991666666666667\\
-10.25	1\\
};
\addlegendentry{Эмперическая функция распределения}

\addplot [color=mycolor2, line width=2.0pt]
  table[row sep=crcr]{%
-14.6	0.0164191802553543\\
-14.56	0.0182616115196319\\
-14.26	0.0384841639513101\\
-14.23	0.041255045223996\\
-14.03	0.0640917772162078\\
-14.01	0.0668325837752475\\
-14.01	0.0668325837752475\\
-13.98	0.0711126981088701\\
-13.87	0.0886196605578353\\
-13.8	0.101322127418623\\
-13.78	0.105184019409264\\
-13.75	0.111174813185738\\
-13.71	0.119537080259893\\
-13.65	0.132896291107367\\
-13.63	0.137569550900824\\
-13.61	0.142353751700467\\
-13.6	0.144787574394957\\
-13.58	0.149738851290379\\
-13.55	0.15737522326591\\
-13.54	0.159976592803275\\
-13.52	0.16526320942507\\
-13.52	0.16526320942507\\
-13.44	0.187524887583811\\
-13.44	0.187524887583811\\
-13.4	0.199319587021773\\
-13.39	0.202336621530021\\
-13.34	0.217826689011598\\
-13.3	0.230696775211326\\
-13.3	0.230696775211326\\
-13.29	0.233979433204215\\
-13.29	0.233979433204215\\
-13.27	0.240621702863458\\
-13.25	0.247365314787773\\
-13.22	0.257667178656179\\
-13.2	0.264656813356656\\
-13.15	0.282542816960121\\
-13.14	0.286188538469964\\
-13.11	0.297257687583698\\
-13.06	0.316127928603262\\
-13.06	0.316127928603262\\
-13.02	0.331580127547079\\
-13	0.33941723065378\\
-12.95	0.359309852210225\\
-12.92	0.371435363342227\\
-12.89	0.383689829066381\\
-12.88	0.387801330161141\\
-12.87	0.391925449461333\\
-12.86	0.396061747668349\\
-12.85	0.400209781367478\\
-12.84	0.404369103169926\\
-12.84	0.404369103169926\\
-12.77	0.433762075086726\\
-12.76	0.437995167398934\\
-12.74	0.44648219491162\\
-12.73	0.450735170100999\\
-12.71	0.459257627187813\\
-12.7	0.463526136861589\\
-12.69	0.467798849660621\\
-12.69	0.467798849660621\\
-12.67	0.476354923032317\\
-12.67	0.476354923032317\\
-12.66	0.48063729976233\\
-12.64	0.489208266362036\\
-12.64	0.489208266362036\\
-12.63	0.493495867231682\\
-12.61	0.502072828112857\\
-12.61	0.502072828112857\\
-12.6	0.506361197046882\\
-12.59	0.510648830948723\\
-12.58	0.514935234573072\\
-12.58	0.514935234573072\\
-12.57	0.519219913101011\\
-12.55	0.527782118751747\\
-12.55	0.527782118751747\\
-12.55	0.527782118751747\\
-12.5	0.5491229638256\\
-12.48	0.55762317623286\\
-12.47	0.561863604470423\\
-12.41	0.587140672658164\\
-12.4	0.591321551517981\\
-12.4	0.591321551517981\\
-12.36	0.607936867697886\\
-12.34	0.616173328797077\\
-12.33	0.620272218714877\\
-12.32	0.624357639017548\\
-12.3	0.632486370334256\\
-12.24	0.656505952204119\\
-12.18	0.679909518480259\\
-12.1	0.710020904596135\\
-12.03	0.735225880962302\\
-12.01	0.742217117801406\\
-11.93	0.769194218997509\\
-11.92	0.772451853662809\\
-11.89	0.782067800286128\\
-11.87	0.788346104545071\\
-11.82	0.803570976902812\\
-11.82	0.803570976902812\\
-11.78	0.815259449197321\\
-11.75	0.823735685300836\\
-11.73	0.829247494868321\\
-11.64	0.852668199568283\\
-11.47	0.8907816587946\\
-11.47	0.8907816587946\\
-11.46	0.892779434173856\\
-11.4	0.904214493660394\\
-11.39	0.906029682629264\\
-11.37	0.909583679745134\\
-11.35	0.913036884866564\\
-11.32	0.918030583210502\\
-11.31	0.919646203761914\\
-11.2	0.935863269914414\\
-11.17	0.939811807525673\\
-11.05	0.953734551566736\\
-11.01	0.957753894897518\\
-10.74	0.978071013813878\\
-10.69	0.980736135766441\\
-10.44	0.990304966227728\\
-10.44	0.990304966227728\\
-10.38	0.991857072757664\\
-10.25	0.994492075232295\\
};
\addlegendentry{Функция распределения}

\end{axis}
\end{tikzpicture}%

\end{figure}

