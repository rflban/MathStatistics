\chapter{Практическая часть}

\lstset{language=matlab}

В листинге~\ref{lst:programm} приведён текст программы.
\begin{lstlisting}[caption={Текст программы},label={lst:programm}]
function lab1()
    X = [
        -13.40, -12.63, -13.65, -14.23, -13.39, -12.36, ...
        -13.52, -13.44, -13.87, -11.82, -12.01, -11.40, ...
        -13.02, -12.61, -13.06, -13.75, -13.55, -14.01, ...
        -11.75, -12.95, -12.59, -13.60, -12.76, -11.05, ...
        -13.15, -13.61, -11.73, -13.00, -12.66, -12.67, ...
        -12.60, -12.47, -13.52, -12.61, -11.93, -13.11, ...
        -13.22, -11.87, -13.44, -12.70, -11.78, -12.30, ...
        -12.89, -13.29, -12.48, -10.44, -12.55, -12.64, ...
        -12.03, -14.60, -14.56, -13.30, -11.32, -12.24, ...
        -11.17, -12.50, -13.25, -12.55, -12.85, -12.67, ...
        -12.41, -12.58, -12.10, -13.54, -12.69, -12.87, ...
        -12.71, -12.77, -13.30, -12.74, -12.73, -12.64, ...
        -12.18, -11.20, -12.40, -13.78, -13.71, -10.74, ...
        -11.89, -13.20, -11.31, -14.26, -10.38, -12.88, ...
        -11.39, -11.35, -12.55, -12.84, -10.25, -12.40, ...
        -14.01, -11.47, -13.14, -12.69, -11.92, -12.86, ...
        -13.06, -12.57, -13.63, -12.34, -12.84, -14.03, ...
        -13.34, -11.64, -13.58, -10.44, -11.37, -11.01, ...
        -13.80, -13.27, -12.32, -10.69, -12.92, -13.29, ...
        -12.58, -13.98, -11.46, -11.82, -12.33, -11.47, ...
    ];
    X = sort(X);

    % Максимальное значение выборки
    M_max = max(X);
    % Минимальное значение выборки
    M_min = min(X);

    % Разброс выборки
    R = M_max - M_min;

    % Оценка мат. ожидания
    M = mean(X);
    % Оценка дисперсии
    D = var(X);

    fprintf('M_max    = %9.5f\n', M_max);
    fprintf('M_min    = %9.5f\n', M_min);
    fprintf('R        = %9.5f\n', R);
    fprintf('mean     = %9.5f\n', M);
    fprintf('varience = %9.5f\n', D);
    fprintf('\n');

    m = floor(log2(length(X))) + 2;
    % Группировка значений в m интервалов
    [N, edges] = histcounts(X, m, 'BinLimits', [M_min, M_max]);

    % Вывод всех интервалов, кроме последнего,
    % потому что последний - это отрезок,
    % а все остальные - полуинтервалы
    fprintf('%d intervals:\n', m);
    for i = 1 : (length(N) - 1)
        fprintf('%3d values in [%f,%f)\n', ...
                N(i), edges(i), edges(i + 1));
    end
    % Вывод последнего отрезка
    fprintf('%3d values in [%f,%f]\n', ...
            N(end), edges(end - 1), edges(end));

    f = normpdf(X, M, sqrt(D));
    F = normcdf(X, M, sqrt(D));

    subplot(2, 1, 1);
    % Построение гистограммы
    histogram(X, m, 'Normalization', 'pdf', ...
              'BinLimits', [M_min, M_max]);
    hold on;
    % Построение графика функции плотности распределения
    plot(X, f, 'LineWidth', 2);
    hold off;

    subplot(2, 1, 2);
    % Построение графика эмпирической функции распределения
    [YY, XX] = ecdf(X);
    stairs(XX, YY, 'LineWidth', 2);
    hold on;
    % Построение графика функции распределения
    plot(X, F, 'LineWidth', 2);
    hold off;
end
\end{lstlisting}

В листинге~\ref{lst:result} приведён результат выполнения описанной программы.
\begin{lstlisting}[language=,numbers=none,caption={Результат программы},label={lst:result}]
M_max    = -10.25000
M_min    = -14.60000
R        =   4.35000
mean     = -12.61483
varience =   0.86533

8 intervals:
  4 values in [-14.600000,-14.056250)
 18 values in [-14.056250,-13.512500)
 20 values in [-13.512500,-12.968750)
 36 values in [-12.968750,-12.425000)
 16 values in [-12.425000,-11.881250)
 14 values in [-11.881250,-11.337500)
  6 values in [-11.337500,-10.793750)
  6 values in [-10.793750,-10.250000]
\end{lstlisting}

\begin{figure}[H]
    \caption{Гистограмма и график функции плотности распределения вероятностей нормальной случайной величины с математическим ожиданием $\hat \mu$ и дисперсией $S^2$}\label{img:plot01}

    % This file was created by matlab2tikz.
%
%The latest updates can be retrieved from
%  http://www.mathworks.com/matlabcentral/fileexchange/22022-matlab2tikz-matlab2tikz
%where you can also make suggestions and rate matlab2tikz.
%
\definecolor{mycolor1}{rgb}{0.00000,0.44700,0.74100}%
\definecolor{mycolor2}{rgb}{0.85000,0.32500,0.09800}%
%
\begin{tikzpicture}[scale=.85]

\begin{axis}[%
width=7.007in,
height=3.401in,
at={(1.175in,5.82in)},
scale only axis,
%xmin=-14.8175,
%xmax=-10.0325,
ymin=0,
ymax=0.6,
xtick={-15, -14.5 ,...,-10},
axis background/.style={fill=white},
legend style={legend cell align=left, align=left, draw=white!15!black}
]
\addplot[ybar interval, fill=mycolor1, fill opacity=0.6, draw=black, area legend] table[row sep=crcr] {%
x	y\\
-14.6	0.0613026819923372\\
-14.05625	0.275862068965517\\
-13.5125	0.306513409961686\\
-12.96875	0.551724137931035\\
-12.425	0.245210727969348\\
-11.88125	0.21455938697318\\
-11.3375	0.0919540229885056\\
-10.79375	0.0919540229885059\\
-10.25	0.0919540229885059\\
};
\addlegendentry{График эмпирической\\плотности распределения}

\addplot [color=mycolor2, line width=2.0pt]
  table[row sep=crcr]{%
-14.6	0.0439935964061537\\
-14.56	0.0481771047163618\\
-14.26	0.0897695338019218\\
-14.23	0.0949890367634914\\
-14.03	0.134821856360976\\
-14.01	0.139272331266005\\
-14.01	0.139272331266005\\
-13.98	0.146098328790479\\
-13.87	0.172574081675708\\
-13.8	0.190476808196545\\
-13.78	0.195721255429795\\
-13.75	0.203683295673108\\
-13.71	0.214458208361271\\
-13.65	0.230896802662044\\
-13.63	0.236433037882949\\
-13.61	0.241990130179128\\
-13.6	0.244775032184411\\
-13.58	0.250354543911841\\
-13.55	0.258738829961034\\
-13.54	0.261535076772332\\
-13.52	0.26712593066571\\
-13.52	0.26712593066571\\
-13.44	0.289369623130366\\
-13.44	0.289369623130366\\
-13.4	0.300342553168853\\
-13.39	0.303062624842424\\
-13.34	0.316487905585523\\
-13.3	0.326974227241384\\
-13.3	0.326974227241384\\
-13.29	0.329554428968055\\
-13.29	0.329554428968055\\
-13.27	0.334660032798846\\
-13.25	0.339687678013619\\
-13.22	0.347070177866876\\
-13.2	0.351877389157436\\
-13.15	0.363452972294885\\
-13.14	0.365686588690256\\
-13.11	0.372211964636193\\
-13.06	0.382462382852282\\
-13.06	0.382462382852282\\
-13.02	0.390053390743035\\
-13	0.393632178940996\\
-12.95	0.4019098437903\\
-12.92	0.406395803716065\\
-12.89	0.410504661622172\\
-12.88	0.41178830638768\\
-12.87	0.413028231774273\\
-12.86	0.414224019147048\\
-12.85	0.415375263958508\\
-12.84	0.416481575976592\\
-12.84	0.416481575976592\\
-12.77	0.422937966802815\\
-12.76	0.42367255502245\\
-12.74	0.424998194247717\\
-12.73	0.425588788944575\\
-12.71	0.426624510617951\\
-12.7	0.427069280083414\\
-12.69	0.427465111451231\\
-12.69	0.427465111451231\\
-12.67	0.428109429266304\\
-12.67	0.428109429266304\\
-12.66	0.428357692721372\\
-12.64	0.428705998869102\\
-12.64	0.428705998869102\\
-12.63	0.428805920876375\\
-12.61	0.428857129606358\\
-12.61	0.428857129606358\\
-12.6	0.428808398577365\\
-12.59	0.42871012741643\\
-12.58	0.428562350185802\\
-12.58	0.428562350185802\\
-12.57	0.428365118097468\\
-12.55	0.427822579756991\\
-12.55	0.427822579756991\\
-12.55	0.427822579756991\\
-12.5	0.425607642499302\\
-12.48	0.424381444853764\\
-12.47	0.423696219998542\\
-12.41	0.418590967455473\\
-12.4	0.417577161721253\\
-12.4	0.417577161721253\\
-12.36	0.413068804749492\\
-12.34	0.41054814961839\\
-12.33	0.409222653744923\\
-12.32	0.407854301988788\\
-12.3	0.404990879995858\\
-12.24	0.395422443544574\\
-12.18	0.38447722091856\\
-12.1	0.367964405343252\\
-12.03	0.351956692550338\\
-12.01	0.347151072346843\\
-11.93	0.32706051655214\\
-11.92	0.324463592592494\\
-11.89	0.316576305404857\\
-11.87	0.311245027281048\\
-11.82	0.297703655052405\\
-11.82	0.297703655052405\\
-11.78	0.286698998743276\\
-11.75	0.278375306241863\\
-11.73	0.272803196525783\\
-11.64	0.24765636990274\\
-11.47	0.201105838620416\\
-11.47	0.201105838620416\\
-11.46	0.198451268323485\\
-11.4	0.182799550806551\\
-11.39	0.180240758596562\\
-11.37	0.17516938566779\\
-11.35	0.170162028666295\\
-11.32	0.162776953107686\\
-11.31	0.160350121076889\\
-11.2	0.134895363702342\\
-11.17	0.128371559494359\\
-11.05	0.104193123827244\\
-11.01	0.0968329401706774\\
-10.74	0.0562680393315699\\
-10.69	0.0504181868892831\\
-10.44	0.0278864913464196\\
-10.44	0.0278864913464196\\
-10.38	0.0239331561143757\\
-10.25	0.0169413605066828\\
};
\addlegendentry{График функции\\плотности распределения}

\end{axis}
\end{tikzpicture}%

\end{figure}

\begin{figure}[H]
    \caption{Графики эмпирической функции распределения и функции распределения нормальной случайной величины с математическим ожиданием $\hat \mu$ и дисперсией $S^2$}\label{img:plot02}
    % This file was created by matlab2tikz.
%
%The latest updates can be retrieved from
%  http://www.mathworks.com/matlabcentral/fileexchange/22022-matlab2tikz-matlab2tikz
%where you can also make suggestions and rate matlab2tikz.
%
\definecolor{mycolor1}{rgb}{0.00000,0.44700,0.74100}%
\definecolor{mycolor2}{rgb}{0.85000,0.32500,0.09800}%
\definecolor{mycolor3}{rgb}{0.92900,0.69400,0.12500}%
\definecolor{mycolor4}{rgb}{0.49400,0.18400,0.55600}%
%
\begin{tikzpicture}[scale=0.8]

\begin{axis}[%
width=7.266in,
height=6.376in,
at={(1.219in,0.861in)},
scale only axis,
xmin=10,
xmax=120,
xlabel style={font=\color{white!15!black}},
xlabel={n},
ymin=0.2,
ymax=2,
xtick={10, 20, ..., 120},
ylabel style={font=\color{white!15!black}},
ylabel={z},
axis background/.style={fill=white},
legend style={legend cell align=left, align=left, draw=white!15!black}
]
\addplot [color=mycolor1]
  table[row sep=crcr]{%
10	0.865332745098039\\
11	0.865332745098039\\
12	0.865332745098039\\
13	0.865332745098039\\
14	0.865332745098039\\
15	0.865332745098039\\
16	0.865332745098039\\
17	0.865332745098039\\
18	0.865332745098039\\
19	0.865332745098039\\
20	0.865332745098039\\
21	0.865332745098039\\
22	0.865332745098039\\
23	0.865332745098039\\
24	0.865332745098039\\
25	0.865332745098039\\
26	0.865332745098039\\
27	0.865332745098039\\
28	0.865332745098039\\
29	0.865332745098039\\
30	0.865332745098039\\
31	0.865332745098039\\
32	0.865332745098039\\
33	0.865332745098039\\
34	0.865332745098039\\
35	0.865332745098039\\
36	0.865332745098039\\
37	0.865332745098039\\
38	0.865332745098039\\
39	0.865332745098039\\
40	0.865332745098039\\
41	0.865332745098039\\
42	0.865332745098039\\
43	0.865332745098039\\
44	0.865332745098039\\
45	0.865332745098039\\
46	0.865332745098039\\
47	0.865332745098039\\
48	0.865332745098039\\
49	0.865332745098039\\
50	0.865332745098039\\
51	0.865332745098039\\
52	0.865332745098039\\
53	0.865332745098039\\
54	0.865332745098039\\
55	0.865332745098039\\
56	0.865332745098039\\
57	0.865332745098039\\
58	0.865332745098039\\
59	0.865332745098039\\
60	0.865332745098039\\
61	0.865332745098039\\
62	0.865332745098039\\
63	0.865332745098039\\
64	0.865332745098039\\
65	0.865332745098039\\
66	0.865332745098039\\
67	0.865332745098039\\
68	0.865332745098039\\
69	0.865332745098039\\
70	0.865332745098039\\
71	0.865332745098039\\
72	0.865332745098039\\
73	0.865332745098039\\
74	0.865332745098039\\
75	0.865332745098039\\
76	0.865332745098039\\
77	0.865332745098039\\
78	0.865332745098039\\
79	0.865332745098039\\
80	0.865332745098039\\
81	0.865332745098039\\
82	0.865332745098039\\
83	0.865332745098039\\
84	0.865332745098039\\
85	0.865332745098039\\
86	0.865332745098039\\
87	0.865332745098039\\
88	0.865332745098039\\
89	0.865332745098039\\
90	0.865332745098039\\
91	0.865332745098039\\
92	0.865332745098039\\
93	0.865332745098039\\
94	0.865332745098039\\
95	0.865332745098039\\
96	0.865332745098039\\
97	0.865332745098039\\
98	0.865332745098039\\
99	0.865332745098039\\
100	0.865332745098039\\
101	0.865332745098039\\
102	0.865332745098039\\
103	0.865332745098039\\
104	0.865332745098039\\
105	0.865332745098039\\
106	0.865332745098039\\
107	0.865332745098039\\
108	0.865332745098039\\
109	0.865332745098039\\
110	0.865332745098039\\
111	0.865332745098039\\
112	0.865332745098039\\
113	0.865332745098039\\
114	0.865332745098039\\
115	0.865332745098039\\
116	0.865332745098039\\
117	0.865332745098039\\
118	0.865332745098039\\
119	0.865332745098039\\
120	0.865332745098039\\
};
\addlegendentry{$\hat S^2(\vec x_N)$}

\addplot [color=mycolor2]
  table[row sep=crcr]{%
10	0.541521111111111\\
11	0.6229\\
12	0.81280606060606\\
13	0.745216666666667\\
14	0.697670879120879\\
15	0.648592380952381\\
16	0.644293333333333\\
17	0.62117794117647\\
18	0.636720261437908\\
19	0.696637426900585\\
20	0.660251578947368\\
21	0.636084761904762\\
22	0.622132683982684\\
23	0.596970355731225\\
24	0.732077536231884\\
25	0.703437666666666\\
26	0.692421538461538\\
27	0.722596866096866\\
28	0.696048015873016\\
29	0.67361724137931\\
30	0.652409655172414\\
31	0.633722795698925\\
32	0.619006350806452\\
33	0.611893371212121\\
34	0.595892335115864\\
35	0.604988739495798\\
36	0.589333968253968\\
37	0.576190840840841\\
38	0.587664366998577\\
39	0.580911470985155\\
40	0.566756346153846\\
41	0.581445975609756\\
42	0.574238617886179\\
43	0.560654706533776\\
44	0.552430179704017\\
45	0.542758585858586\\
46	0.655104396135266\\
47	0.64199065679926\\
48	0.628712765957447\\
49	0.626865731292517\\
50	0.681979959183674\\
51	0.729478588235294\\
52	0.719444457013575\\
53	0.749080188679246\\
54	0.740947344514325\\
55	0.775456565656566\\
56	0.76265038961039\\
57	0.753170551378446\\
58	0.740812946158499\\
59	0.728151607247224\\
60	0.715977853107345\\
61	0.706153770491803\\
62	0.695116367001586\\
63	0.690815719406042\\
64	0.689617633928571\\
65	0.678921490384615\\
66	0.668657832167832\\
67	0.658567435549525\\
68	0.648739135206321\\
69	0.643399658994032\\
70	0.634087391304348\\
71	0.625050422535211\\
72	0.616476056338028\\
73	0.612628652968036\\
74	0.637065235098112\\
75	0.629975603603603\\
76	0.635997894736842\\
77	0.639677204374573\\
78	0.683648901098901\\
79	0.683889581304771\\
80	0.678079430379747\\
81	0.694453456790123\\
82	0.715132083709726\\
83	0.772954011166618\\
84	0.764019320137693\\
85	0.775234537815126\\
86	0.786945567715458\\
87	0.777968671478214\\
88	0.769349164054336\\
89	0.826594330949949\\
90	0.817980137328339\\
91	0.829413162393162\\
92	0.83565134973722\\
93	0.829197078073866\\
94	0.820297277510867\\
95	0.817199820828667\\
96	0.809085614035088\\
97	0.802426030927835\\
98	0.794219072164948\\
99	0.795829952587095\\
100	0.788810101010101\\
101	0.781257207920792\\
102	0.791981527858668\\
103	0.788558690272226\\
104	0.791258168409261\\
105	0.791574963369963\\
106	0.831223189577718\\
107	0.838822535707988\\
108	0.855685661128418\\
109	0.86035915732246\\
110	0.856085704753961\\
111	0.849253087633088\\
112	0.875615822072072\\
113	0.868571223135272\\
114	0.864740723490141\\
115	0.857179527078566\\
116	0.8653823988006\\
117	0.869903183023873\\
118	0.868081211067652\\
119	0.861465588947444\\
120	0.865332745098039\\
};
\addlegendentry{$\hat S^2(\vec x_n)$}

\addplot [color=mycolor3]
  table[row sep=crcr]{%
10	0.288060550341353\\
11	0.340251655230794\\
12	0.45442460738269\\
13	0.425310106816267\\
14	0.405585736925727\\
15	0.383380761791958\\
16	0.386641108201628\\
17	0.377957143061275\\
18	0.392365992727796\\
19	0.434353237924762\\
20	0.416168284302517\\
21	0.405014961149502\\
22	0.399894615478748\\
23	0.387135304749173\\
24	0.478720639798833\\
25	0.463613642349946\\
26	0.459744924137553\\
27	0.483154211763858\\
28	0.468505695472828\\
29	0.456279355613045\\
30	0.444577726660581\\
31	0.434324723179688\\
32	0.426565531697866\\
33	0.423874916107105\\
34	0.414862768280063\\
35	0.42322253538413\\
36	0.414175157503766\\
37	0.406735226963126\\
38	0.416605004172795\\
39	0.413510150129268\\
40	0.405031980692383\\
41	0.417117527686894\\
42	0.413466742667926\\
43	0.405124946819884\\
44	0.400558026251427\\
45	0.394858262291948\\
46	0.478130048038718\\
47	0.470026239525137\\
48	0.461702915612263\\
49	0.461703239170968\\
50	0.503733774696365\\
51	0.540316034903334\\
52	0.534324225825886\\
53	0.557797003869771\\
54	0.553152547055332\\
55	0.580357422301474\\
56	0.572158193694558\\
57	0.566382436470075\\
58	0.558374039712328\\
59	0.550065142323687\\
60	0.542055811647171\\
61	0.535763586794687\\
62	0.528492953867723\\
63	0.526296882679057\\
64	0.526433797759715\\
65	0.519281027617177\\
66	0.512407785108283\\
67	0.505618537145327\\
68	0.498983859513743\\
69	0.495763119516068\\
70	0.489444443601213\\
71	0.483297661854106\\
72	0.477470118911101\\
73	0.475272977630311\\
74	0.495030007444584\\
75	0.49029738706951\\
76	0.495754443920272\\
77	0.499383491948614\\
78	0.534510787757009\\
79	0.535485170789528\\
80	0.531702326707487\\
81	0.54531373573894\\
82	0.562333511776205\\
83	0.608632579890785\\
84	0.602406231456155\\
85	0.612056947760912\\
86	0.622110213729051\\
87	0.615799385667244\\
88	0.609741818561214\\
89	0.655920749446683\\
90	0.649874502178668\\
91	0.659746399662435\\
92	0.66549130890332\\
93	0.661116823036643\\
94	0.654767497064925\\
95	0.653028180867232\\
96	0.647259706192188\\
97	0.642631979192249\\
98	0.636742514865004\\
99	0.638709215290475\\
100	0.633735521664745\\
101	0.62831263401675\\
102	0.637582831753078\\
103	0.635461407854625\\
104	0.638264804270647\\
105	0.639140516116905\\
106	0.671796555859987\\
107	0.678578945640435\\
108	0.692865900763751\\
109	0.69729083969207\\
110	0.694457029625561\\
111	0.689531421261035\\
112	0.711564492307129\\
113	0.706455648777074\\
114	0.703945987918781\\
115	0.698384259637339\\
116	0.705659679001495\\
117	0.709934399059781\\
118	0.709027807175233\\
119	0.704193651975309\\
120	0.707920202437793\\
};
\addlegendentry{$\underline{\sigma}^2(\vec x_n)$}

\addplot [color=mycolor4]
  table[row sep=crcr]{%
10	1.46572168525412\\
11	1.58084444475329\\
12	1.95436764555008\\
13	1.71116523928196\\
14	1.53936358828262\\
15	1.38195141424851\\
16	1.33101151809903\\
17	1.24834080699404\\
18	1.24821768464206\\
19	1.33534249154733\\
20	1.23996874579649\\
21	1.17241879671005\\
22	1.12711952004286\\
23	1.06446201228061\\
24	1.28625836168805\\
25	1.21909198821361\\
26	1.18472763806558\\
27	1.22162216223004\\
28	1.16357103766745\\
29	1.11421443678693\\
30	1.06841477099962\\
31	1.02806642535224\\
32	0.995260944517841\\
33	0.975521736548525\\
34	0.94239163377942\\
35	0.949471501767395\\
36	0.918169370734058\\
37	0.891452955070385\\
38	0.903162345154353\\
39	0.887104995959514\\
40	0.860212557827718\\
41	0.877346298083442\\
42	0.861603227264651\\
43	0.836677667056909\\
44	0.820118421944042\\
45	0.801725426861385\\
46	0.963003014115295\\
47	0.939329325325113\\
48	0.915763189194204\\
49	0.909102807138723\\
50	0.984872296044173\\
51	1.04917918969832\\
52	1.0306687511896\\
53	1.06902517055425\\
54	1.05350150753824\\
55	1.09860465371497\\
56	1.07669138219135\\
57	1.05970344138941\\
58	1.03888224972412\\
59	1.01785499752981\\
60	0.997718092194705\\
61	0.981042580992713\\
62	0.962855251933958\\
63	0.954143601316492\\
64	0.949816423058342\\
65	0.932526317522709\\
66	0.915977719494446\\
67	0.899805624226384\\
68	0.884123574855455\\
69	0.874669587988027\\
70	0.859919035831105\\
71	0.845654041327347\\
72	0.832120492837997\\
73	0.825053090069623\\
74	0.856060661065672\\
75	0.844697303309705\\
76	0.850961226462219\\
77	0.854104428023181\\
78	0.910956954034611\\
79	0.909459504246865\\
80	0.899969935655448\\
81	0.919935693338067\\
82	0.945548434972826\\
83	1.02011737011497\\
84	1.00650301375068\\
85	1.01946638242981\\
86	1.0330655969469\\
87	1.01953617168605\\
88	1.00654880930042\\
89	1.07966170351318\\
90	1.06668126659968\\
91	1.07987084106932\\
92	1.08629300697231\\
93	1.07624774314281\\
94	1.06308921610072\\
95	1.05750314874797\\
96	1.04547477739044\\
97	1.0353809806185\\
98	1.02334416970897\\
99	1.02399489767176\\
100	1.01357453510037\\
101	1.00251837513123\\
102	1.01493342036213\\
103	1.00922873633475\\
104	1.01138274484088\\
105	1.01050758440497\\
106	1.0597992574311\\
107	1.06817526661168\\
108	1.08833097845044\\
109	1.09297060741396\\
110	1.08626396540798\\
111	1.07634621096501\\
112	1.10849132629864\\
113	1.09833529523173\\
114	1.09227757652464\\
115	1.081541384835\\
116	1.09071214729329\\
117	1.09524202659795\\
118	1.09179935220178\\
119	1.08235512330899\\
120	1.08610118368759\\
};
\addlegendentry{$\overline{\sigma}^2(\vec x_n)$}

\end{axis}
\end{tikzpicture}%

\end{figure}

