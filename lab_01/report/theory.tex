\chapter{Теоретическая часть}
Пусть $\vec x = \big( x_1, \dots, x_n \big)$~--- выборка из генеральной совокупности $X$.

Тогда:
\begin{enumerate}[label=\arabic*$\,^\circ$]
\item Максимальное значение выборки:
\begin{flalign*}
    &
    M_{\max} = \max \{ x_1, \dots, x_n \}
    &
\end{flalign*}

\item Минимальное значение выборки:
\begin{flalign*}
    &
    M_{\min} = \min \{ x_1, \dots, x_n \}
    &
\end{flalign*}

\item Размах выборки:
\begin{flalign*}
    &
    R = M_{\max} - M_{\min}
    &
\end{flalign*}

\item Выборочное среднее:
\begin{flalign*}
    &
    \hat \mu = \overline x = \frac{1}{n} \sum_{i=1}^{n} x_i
    &
\end{flalign*}

\item Состоятельная оценка дисперсии:
\begin{flalign*}
    &
    S^2 = \frac{1}{n-1} \sum_{i=1}^n (x_i - \overline x){}^2
    &
\end{flalign*}
\end{enumerate}

$\blacksquare{}$

Пусть $\vec x$~--- выборка из генеральной совокупности $X$,

\begin{equation*}
    \vec x = (x_1, \ldots, x_n)
\end{equation*}

Расположим значения $x_1, \ldots, x_n$ в порядке неубывания

\begin{equation*}
    x_{(1)} \le x_{(2)} \le \ldots \le x_{(n)}
\end{equation*}
где $x_{(i)}$~--- $i$-й элемент полученной последовательности. Такая последовательность называется вариационным рядом.

Если объем $n$ статистической выборки $\vec x$ велик ($n \geq 50$), то можно сгруппировать выборку в интервальный статистический ряд. Для этого отрезок $J = \big[ x_{(1)}, x_{(n)} \big]$ делят на $p = [\log_2n] + 1$ (где $[a]$~--- целая часть числа $a$) равновеликих частей:

\begin{equation*}
    J_i = \big[ a_{i-1}, a_i \big), i = \overline{1, p-1} \\
\end{equation*}

\begin{equation*}
    J_{p} = \big[ a_{p-1}, a_p \big]
\end{equation*}
где
\begin{equation*}
    a_i = x_{(1)} + i\Delta, i = \overline{0, p}
\end{equation*}

\begin{equation*}
    \Delta = \frac{|J|}{p} = \frac{x_{(n)} - x_{(1)}}{p}
\end{equation*}

Интервальным статистическим рядом называют таблицу

\begin{table}[H]
    \centering
    \begin{tabular}{ccccc}
        \toprule
        $J_1$ & $\dots$ & $J_i$ & $\dots$ & $J_p$ \\
        \midrule
        $n_1$ & $\dots$ & $n_i$ & $\dots$ & $n_p$ \\
        \bottomrule
    \end{tabular}
\end{table}
где $n_i$~--- количество элементов $\vec x$, которые $\in J_i$.

Предположим, что для выборки $\vec x$ построен интервальный статистический
ряд

\begin{equation*}
    \big( J_i, n_i \big), i = \overline{1; p}
\end{equation*}

\textbf{Эмпирической плотностью} (отвечающей выборке $\vec x$) называют функцию
\begin{equation*}
    \hat f_n(x) =
    \begin{cases}
        \frac{n_i}{n \Delta}, x \in J_i, i = \overline{1; p} \\
        0, \text{ иначе} \\
    \end{cases}
\end{equation*}

\textbf{Гистограммой} называют график эмпирической плотности.

$\blacksquare{}$

Для выборки $\vec x$ обозначим $n(x, \vec x)$~--- число элементов вектора $\vec x$, которые имеют значения меньше $x$.

\textbf{Эмпирической функцией распределения} называют функцию
\begin{equation*}
    F_n : \mathbb{R} \to \mathbb{R}
\end{equation*}
определенную условием

\begin{equation*}
    F_n(x) = \frac{n(x, \vec x)}{n}
\end{equation*}

$\blacksquare{}$

