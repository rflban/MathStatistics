\section*{Задача \textnumero2\\\textit{Метод моментов}}

\subsection*{Условие}
\sloppy С использованием метода моментов для случайной выборки $\vec{X} = (X_1,\dots,X_n)$ из генеральной совокупности $X$ найти точечные оценки указанных параметров заданного закона распределения:
\begin{equation*}
f_X(x) = 3\theta{}x^2e^{-\theta{}x^3},\quad{}x > 0
\end{equation*}

\subsection*{Решение}
Данный закон распределения зависит от единственного параметра $\theta$, следовательно система метода моментов будет содержать только одно уравнение с одной неизвестной. Это уравнение имеет вид:
\begin{equation} \label{eq:equation1}
    MX^1 = \hat{\mu}_1(\overrightarrow{X})
\end{equation}
При этом:
\begin{equation} \label{eq:equation2}
    \hat{\mu}_1(\overrightarrow{X}) = \overline{X}
\end{equation}

Теперь необходимо вычислить математическое ожидание случайной величины, подчиняющейся данному закону распределения:
\begin{equation*}
    MX = \int^{+\infty}_{-\infty}x \cdot f_X(x)dx
\end{equation*}

\begin{flalign}
    &
        MX =
        \int^{+\infty}_0 3 \theta x^3 e^{-\theta x^3} dx =
    \int^{+\infty}_0xe^{-\theta x^3} d(\theta x^3) =
    \nonumber &\\
    & =
        \frac{1}{\sqrt[3]{\theta}}
        \int^{+\infty}_0 (\theta x^3){}^\frac{1}{3} e^{-\theta x^3} d(\theta x^3) =
        \begin{bmatrix}
            t = x^3
            x = t^\frac{1}{3}
        \end{bmatrix} =
        \frac{1}{\sqrt[3]{\theta}}
        \int^{+\infty}_0 t^\frac{1}{3} e^{-t} dt
    \label{eq:mxint} &
\end{flalign}
\\

Обратим внимание на интегральное определение гамма-функции:
\begin{equation} \label{eq:gamma}
    \Gamma(z) = \int^{+\infty}_0 t^{z - 1} e^{-t} dt
\end{equation}

Заметим схожесть в~\ref{eq:mxint} и~\ref{eq:gamma}. Таким образом:
\begin{equation} \label{eq:mx}
    MX = \frac{\Gamma(\frac{4}{3})}{\sqrt[3]{\theta}}
\end{equation}

Учитывая~\ref{eq:equation1},~\ref{eq:equation2} и~\ref{eq:mx}, имеем:
\begin{equation*}
    \frac{\Gamma(\frac{4}{3})}{\sqrt[4]{\theta}} = \overline{X} \Rightarrow
    \underline{\underline{\theta = \Big(\frac{\Gamma(\frac{4}{3})}{\overline{X}}\Big){}^3}}
\end{equation*}

