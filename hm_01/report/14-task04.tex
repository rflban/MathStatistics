\section*{Задача \textnumero4\\\textit{Доверительные интервалы}}

\subsection*{Условие}
\sloppy Плотность распределения времени $t$ безотказной работы радиоэлектронной аппаратуры между двумя последовательными отказами дается формулой $f(t) = (1/T)e^{-(t/T)}$, $t \geqslant 0$. Для оценки параметра $T$ провели испытания аппаратуры до появления $d = 5$ отказов. Общая продолжительность $S$ работы с начала испытания до последнего отказа оказалась равной 1600 ч. Определить границы 80\%-го доверительного интервала для параметра $T$, если известно, что величина $2S/T$ распределена по закону $\chi^2(2d)$.

\subsection*{Решение}
Пусть $q_{\alpha}$, $q_{1 - \alpha}$~--- квантили соответствующих уровней случайной величины $2S/T$, распределенной по закону $\chi^2$ с степенями свободы $n = 2d = 10$. При этом $2\alpha + \gamma = 1 \Rightarrow \alpha = \frac{1 - \gamma}{2} = \frac{1 - 0.8}{2} = 0.1$. Таким образом:
\begin{equation*}
    \gamma = P\Big\{q_{\alpha} < \frac{2S}{T} < q_{1 - \alpha}\Big\}
\end{equation*}
Из полученного тождества получим оценку параметра $T$:
\begin{equation*}
    \gamma = P\Big\{\frac{2S}{q_{1 - \alpha}} < T < \frac{2S}{q_{\alpha}}\Big\}
\end{equation*}

Значение квантилей $q_{\alpha}$, $q_{1 - \alpha}$ можем получить при помощи функции chi2inv пакета MATLAB\@:
\begin{equation*}
    q_{0.1} = 4.8652,\quad q_{0.9} = 15.9872
\end{equation*}

Имеем итоговую оценку:
\begin{equation*}
    \underline{\underline{0.8 \approx P\Big\{200.16 < T < 657.73\Big\}}}
\end{equation*}

