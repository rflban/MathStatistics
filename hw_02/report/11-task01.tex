\section*{Задача \textnumero1\\\textit{Проверка параметрических гипотез}}

\subsection*{Условие}
Расстояние между двумя подвижными объектами определяется с помощью гамма-дальномера, точность которого характеризуется средним квадратичным отклонением $\sigma = 10$м. С интервалом 12 минут проведено 2 серии измерений, в результате чего получены значения $\overline{x}_{n_1} = 832$м, $n_1 = 5$, $\overline{y}_{n_2} = 840$м, $n_2 = 3$. Предполагая, что ошибка измерений подчиняется нормальному закону, при уровне значимости $\alpha = 0.05$ проверить гипотезу о том, что за указанное время расстояние между объектами не увеличилось.

\subsection*{Решение}
Пусть случайная величина $X$~--- результат первой серии измерений, а $Y$~--- второй. По условию $X \sim N(m_1, \sigma),\ Y \sim N(m_2, \sigma)$, где $\sigma = 10,\ m_1 = MX,\ m_2 = MY$.

Введём две гипотезы:
\begin{equation*}
    H_0 = \big\{\text{Расстояние не изменилось}\big\} = \big\{m_2 = m_1\big\}
\end{equation*}
\begin{equation*}
    H_1 = \big\{\text{Расстояние увеличилось}\big\} = \big\{m_2 > m_1\big\}
\end{equation*}

Гипотеза $H_0$ является основной, тогда как $H_1$~--- конкурирующей. Для рассматриваемого случая используем статистику
\begin{equation*}
    T(\overrightarrow{Y}_{n_2}, \overrightarrow{X}_{n_1}) = \frac{\overline{Y} - \overline{X}}{\sqrt{\frac{\sigma_2^2}{n_2} + \frac{\sigma_1^2}{n_1}}} \sim N(0, 1)
\end{equation*}

Так же имеем определяющее критическую область условие:
\begin{equation*}
    W = \Big\{\big(\vec y, \vec x\big):\ T\big(\overrightarrow{Y}_{n_2}, \overrightarrow{X}_{n_1}\big) \geqslant t_{1 - \alpha}\Big\}
\end{equation*}

Тогда
\begin{flalign*}
    &
    t_{1 - \alpha} = t_{0.95} = 1.645
    \\ &
    T(\vec x, \vec y) = \frac{840 - 832}{\sqrt{\frac{100}{3} + \frac{100}{5}}} = \frac{8}{\sqrt{\frac{800}{15}}} \approx 1.0954
    &
\end{flalign*}

\sloppy Таким образом, $1.0954 \ngeqslant 1.645 \Rightarrow$ условие не выполнено, принимается гипотеза $H_0$, а следовательно:
\newline
\underline{\underline{За указанное время расстояние между объектами не увеличилось}}.



%Условие данной задачи описывает схему испытаний Бернули. При этом $n = 100$~--- число испытаний. Очевидно, что $n \gg 1$. Тогда по интегральной теореме Муавра-Лапласа имеем:
%\begin{equation*}
    %P\Big\{k_1 \leqslant{} k \leqslant k_2\Big\} \approx \Phi(x_2) - \Phi(x_1)
%\end{equation*}
%где, $k$~--- число успехов; $x_i = \frac{k_i - np}{\sqrt{npq}}, i = \overline{1,2}$; $p = 0.9$~--- вероятность успеха, $q = 1 - p = 0.1$~--- вероятность неудачи.
%\begin{flalign*}
    %&
    %P\Big\{x \in [85, 95]\Big\} = \Phi\bigg(\frac{95-90}{3}\bigg) - \Phi\bigg(\frac{85-90}{3}\bigg) \approx 2\cdot\Phi(1.67) \approx \underline{\underline{0.905}}
    %&
%\end{flalign*}

