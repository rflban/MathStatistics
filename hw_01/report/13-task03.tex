\section*{Задача \textnumero3\\\textit{Метод максимального правдоподобия}}

\subsection*{Условие}
\sloppy С использованием метода максимального правдоподобия для случайной выборки $\vec{X} = (X_1,\dots,X_n)$ из генеральной совокупности $X$ найти точечные оценки параметров заданного закона распределения:
\begin{equation*}
    f_X(x) = \frac{\theta}{2} e^{-\theta |x - 2|}
\end{equation*}
А так же вычислить выборочные значения найденных оценок для выборки:
\begin{equation*}
    \vec{x}_5 = (-2, 4, -3, 5, 1)
\end{equation*}

\subsection*{Решение}
Рассматриваемая случайная величина является непрерывной. Следовательно, функция правдоподобия принимает вид:
\begin{equation*}
    \mathcal{L}(\vec{x}_5) = f_X(x1) \cdot f_X(x2) \cdot f_X(x3) \cdot f_X(x4) \cdot f_X(x5) = \frac{\theta^5}{32}e^{-15 \theta}
\end{equation*}

Для упрощения вычислений, проинтегрируем полученную функцию:
\begin{equation*}
    \ln \mathcal{L} = \ln\frac{\theta^5}{32} + \ln e^{-15 \theta} = 5 \ln \theta - \ln 32 - 15 \theta
\end{equation*}

Имея необходимое условие экстремума $\frac{\partial \ln L}{\partial \theta} = 0$, найдем искомый параметр:
\begin{equation*}
    \frac{5}{\theta} - 15 = 0 \Rightarrow \theta = \frac{1}{3}
\end{equation*}

Зная достаточное условие экстремума $\frac{\partial^2 \ln \mathcal{L}}{\partial \theta^2} \neq 0$, проверим полученное значение:
\begin{equation*}
    -\frac{5}{\theta^2} \neq 0 \Rightarrow \underline{\underline{\theta = \frac{1}{3}}} \text{~--- достаточное условие выполняется}
\end{equation*}

