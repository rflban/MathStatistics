\section*{Задача \textnumero1\\\textit{Предельные теоремы теории вероятностей}}

\subsection*{Условие}
\sloppy Стрелок поражает мишень с вероятностью 0.9. Какова вероятность того, что при 100 выстрелах число попаданий будет не менее 85 и не более 95?

\subsection*{Решение}
Условие данной задачи описывает схему испытаний Бернули. При этом $n = 100$~--- число испытаний. Очевидно, что $n \gg 1$. Тогда по интегральной теореме Муавра-Лапласа имеем:
\begin{equation*}
    P\Big\{k_1 \leqslant{} k \leqslant k_2\Big\} \approx \Phi(x_2) - \Phi(x_1)
\end{equation*}
где, $k$~--- число успехов; $x_i = \frac{k_i - np}{\sqrt{npq}}, i = \overline{1,2}$; $p = 0.9$~--- вероятность успеха, $q = 1 - p = 0.1$~--- вероятность неудачи.
\begin{flalign*}
    &
    P\Big\{x \in [85, 95]\Big\} = \Phi\bigg(\frac{95-90}{3}\bigg) - \Phi\bigg(\frac{85-90}{3}\bigg) \approx 2\cdot\Phi(1.67) \approx \underline{\underline{0.905}}
    &
\end{flalign*}

